\chapter{libft-\/extended}
\hypertarget{md__2_users_2thibault_2kdrive_21-_p_r_o_j_e_c_t_s_2_p-42_2mini_r_t_2mini_r_t__example_2lib_2libft_2_r_e_a_d_m_e}{}\label{md__2_users_2thibault_2kdrive_21-_p_r_o_j_e_c_t_s_2_p-42_2mini_r_t_2mini_r_t__example_2lib_2libft_2_r_e_a_d_m_e}\index{libft-\/extended@{libft-\/extended}}
\label{md__2_users_2thibault_2kdrive_21-_p_r_o_j_e_c_t_s_2_p-42_2mini_r_t_2mini_r_t__example_2lib_2libft_2_r_e_a_d_m_e_autotoc_md41}%
\Hypertarget{md__2_users_2thibault_2kdrive_21-_p_r_o_j_e_c_t_s_2_p-42_2mini_r_t_2mini_r_t__example_2lib_2libft_2_r_e_a_d_m_e_autotoc_md41}%
